\documentclass[11pt]{article}

\usepackage{amsmath,amssymb}
\usepackage{hyperref}
\usepackage{geometry}
\geometry{margin=1in}

\title{ISR as a Normalization-Resistant Polynomial Regime}
\author{Inacio F. Vasquez}
\date{\today}

\begin{document}
\maketitle

\begin{abstract}
We exhibit an executable computational regime, the Implicit State Reduction (ISR) model, which runs in polynomial time yet resists normalization into transcript-compressible refinement systems. The regime admits fast local evolution while provably failing to admit bounded-information summaries compatible with URF-style normalization. This provides a concrete separation between runtime efficiency and normalization admissibility.
\end{abstract}

\section{Motivation}
Most known polynomial-time computational models admit some form of transcript normalization: their executions can be compressed into bounded-width, locality-respecting refinement processes without loss of correctness. The Unified Rigidity Framework (URF) formalizes this intuition and identifies entropy-based barriers governing admissible refinement.

ISR is constructed to violate this expectation. Although each step is polynomial-time and locally defined, the induced state evolution exhibits collision patterns that cannot be normalized without exceeding transcript capacity bounds.

\section{The ISR Model}
An ISR system maintains an implicit global state represented abstractly by a compressed decision structure. Updates are driven by local constraints (clauses, XOR relations), but the state evolution includes controlled non-linear mixing that preserves runtime efficiency while destroying normalization invariants.

Formally, ISR maintains:
\begin{itemize}
\item an implicit state magnitude $S_t$,
\item a trace of structural summaries $(\text{count}_t, \text{nodes}_t)$,
\item deterministic update rules parameterized by local constraints.
\end{itemize}

All updates run in polynomial time in the number of variables.

\section{Barrier Property}
Define $P(S_t)$ to be an implicit dimension proxy extracted from the ISR state. Empirically and structurally, $P$ satisfies:
\begin{itemize}
\item monotonicity under URF-admissible refinements,
\item bounded per-step entropy loss for normalized systems,
\item super-polynomial growth along ISR traces.
\end{itemize}

This violates the URF per-step entropy loss bound and blocks normalization.

\section{Normalization Failure}
We implement a normalization attempt harness that:
\begin{enumerate}
\item records ISR traces,
\item attempts compression into URF transcripts under fixed bit budgets,
\item detects invariant-preserving collisions.
\end{enumerate}

The harness deterministically fails on ISR instances, emitting a signed failure certificate. The failure persists across seeds and parameter scales.

\section{Separation Result}
ISR demonstrates a strict separation:
\[
\text{Polynomial-time execution} \;\;\not\Rightarrow\;\; \text{Normalization admissibility}.
\]

This shows that runtime efficiency alone does not guarantee compatibility with refinement-based complexity analyses.

\section{Status and Artifacts}
The ISR regime is released as an executable, sealed artifact with:
\begin{itemize}
\item deterministic core model,
\item regression tests and CI verification,
\item normalization-failure certificates,
\item Zenodo DOI and immutable release tag.
\end{itemize}

All artifacts are publicly available at
\[
\href{https://github.com/inaciovasquez2020/ncr-new-computational-regime}{github.com/inaciovasquez2020/ncr-new-computational-regime}.
\]

\section{Outlook}
Remaining work includes automated extraction of minimal divergence indices and integration of ISR as a formal corollary within the Chronos / URF hierarchy.

\end{document}

